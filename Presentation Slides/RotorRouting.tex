\documentclass{beamer}
 
\usepackage[utf8]{inputenc}
\usepackage{amssymb}
\usepackage{amsthm}
\usepackage{etoolbox}
\usepackage{multicol}
\usepackage{graphicx}
\usepackage{tikz}

\makeatletter
\AtBeginEnvironment{proof}{\let\@addpunct\@gobble}
\makeatother

\newtheorem{conj}{Conjecture}
\newtheorem{cor}[theorem]{Corollary}
\newtheorem{defn}{Definition}
\newtheorem{observe}{Observation}
\renewcommand{\thefootnote}{\arabic{footnote}}

\usetheme{Madrid}
\usecolortheme{beaver}
%---------------------------------------------------------------------------------
\title{Rotor routing on 2-manifolds}
\author{Taro~Shima \and Nicholas~Tomlin}
\institute{Brown University}
\date{SUMS, March 2016}
%---------------------------------------------------------------------------------
\setbeamertemplate{footline} {
	\leavevmode%
	\hbox{%
		\begin{beamercolorbox}[wd=.33\paperwidth,ht=2.25ex,dp=1ex,center]   {author in head/foot}%
			\usebeamerfont{author in head/foot}\insertshortauthor
		\end{beamercolorbox}%
		\begin{beamercolorbox}[wd=.34\paperwidth,ht=2.25ex,dp=1ex,center]{title in head/foot}%
			\usebeamerfont{title in head/foot}\insertshorttitle
		\end{beamercolorbox}%
		\begin{beamercolorbox}[wd=.33\paperwidth,ht=2.25ex,dp=1ex,right]{date in head/foot}%
			\insertframenumber{} / \inserttotalframenumber\hspace*{2ex} 
		\end{beamercolorbox}}%
	\vskip0pt%
}
%---------------------------------------------------------------------------------

\begin{document}

%---------------------------------------------------------------------------------
%------------------------------TITLE SLIDE---------------------------------
%---------------------------------------------------------------------------------

\frame{\titlepage}

%---------------------------------------------------------------------------------
%-----------------------------BLANK INTRO--------------------------------
%---------------------------------------------------------------------------------

%\begin{frame}
%	\frametitle{Outline}
	
%\end{frame}

%---------------------------------------------------------------------------------
%-------------------------------OVERVIEW----------------------------------
%---------------------------------------------------------------------------------

\begin{frame}
	\frametitle{Outline}
	\pause
	\begin{itemize}
		\item Definition of rotor routing\pause
		\item Rotor routing on $\mathbb{Z}^2$\pause
		\item Recurrence and transience\pause
		\item Rotor routing on other 2-manifolds\pause
		\item Visualizations and experimental results
	\end{itemize}
	
\end{frame}

%---------------------------------------------------------------------------------
%-----------------------------BLANK INTRO--------------------------------
%---------------------------------------------------------------------------------

%\begin{frame}
%	\frametitle{What is rotor routing?}
	
%\end{frame}

%---------------------------------------------------------------------------------
%-----------------------------MORAL INTRO-------------------------------
%---------------------------------------------------------------------------------

\begin{frame}
	\frametitle{What is rotor routing?}
	
	\pause
	
	Rotor routing is a deterministic version of the \underline{random walk}.

\end{frame}

%---------------------------------------------------------------------------------
%--------------------------ROTOR DEFINITION---------------------------
%---------------------------------------------------------------------------------

\begin{frame}
	\frametitle{What is rotor routing?}
	
  	 \begin{defn}[Rotor]
    		An arrow pointing from one vertex to another.\\
    	\end{defn}
	
    	\begin{center}
    		\begin{tikzpicture}
    			\def \n {2}
    			\def \radius {3cm}
    			\def \margin {8}
			\foreach \s in {1,...,\n} {
        				\node[draw, circle] at ({360/\n * (\s - 1)}:\radius) {$\s$};
     				 \draw [->] (-2.5,0) -- (2.5,0);
			}
    		\end{tikzpicture}
	\end{center}
	
	\pause 
	
	\begin{defn}[Rotor Configuration]
		An assignment of a single rotor to every vertex on a graph.
	\end{defn}
	
\end{frame}

%---------------------------------------------------------------------------------
%------------------CONFIGURATION DEFINITION---------------------
%---------------------------------------------------------------------------------

\begin{frame}
	\frametitle{What is rotor routing?}
	
	\begin{defn}[Rotor Configuration]
		An assignment of a single rotor to every vertex on a graph.
	\end{defn}
	
%\end{frame}

%---------------------------------------------------------------------------------
%---------------------MECHANISM DEFINITION------------------------
%---------------------------------------------------------------------------------

%\begin{frame}
%	\frametitle{What is rotor routing?}
\pause	

	\begin{defn}[Rotor Mechanism]
    		Given a set of possible rotors $\mathcal{E}$ starting at a given point, a rotor mechanism is a cyclic permutation of $\mathcal{E}$.
	\end{defn}
	
%\end{frame}

%---------------------------------------------------------------------------------
%---------------------ROTOR WALK DEFINITION----------------------
%---------------------------------------------------------------------------------
\pause
%\begin{frame}
	\frametitle{What is rotor routing?}
	
	\begin{defn}[Rotor Walk]
		A traversal of a particle along a initial rotor configuration. The particle follows the direction of rotors, which then change direction according to the rotor mechanism.
    	\end{defn}

\end{frame}

%---------------------------------------------------------------------------------
%-------------------------------EXAMPLE 1---------------------------------
%---------------------------------------------------------------------------------

\begin{frame}
	\frametitle{Rotor walk example}
	
	\begin{figure}
		\begin{tikzpicture}
			\def \n {3}
			\def \radius {3cm}
			\def \margin {8}
			\foreach \s in {1,2} {
				\node[draw, circle] at ({360/\n * (\s - 1)}:\radius) {$\s$};
				\draw[->, >=latex] ({360/\n * (\s - 1)+\margin}:\radius)
				 	arc ({360/\n * (\s - 1)+\margin}:{360/\n * (\s)-\margin}:\radius);
			}
			\foreach \s in {3} {
				\node[draw, circle, fill=black] at ({360/\n * (\s - 1)}:\radius) {\color{white} $\s$};
				\draw[->, >=latex] ({360/\n * (\s - 1)+\margin}:\radius)
				 	arc ({360/\n * (\s - 1)+\margin}:{360/\n * (\s)-\margin}:\radius);
			}
		\end{tikzpicture}
	\end{figure}
	
\end{frame}

%---------------------------------------------------------------------------------
%-------------------------------EXAMPLE 2---------------------------------
%---------------------------------------------------------------------------------

\begin{frame}
	\frametitle{Rotor walk example}
	
	\begin{figure}
		\begin{tikzpicture}
			\def \n {3}
			\def \radius {3cm}
			\def \margin {8}
			\foreach \s in {1} {
				\node[draw, circle, fill=black] at ({360/\n * (\s - 1)}:\radius) {\color{white} $\s$};
				\draw[->, >=latex] ({360/\n * (\s - 1)+\margin}:\radius)
				 	arc ({360/\n * (\s - 1)+\margin}:{360/\n * (\s)-\margin}:\radius);
			}
			\foreach \s in {2} {
				\node[draw, circle] at ({360/\n * (\s - 1)}:\radius) {$\s$};
				\draw[<->, >=latex] ({360/\n * (\s - 1)+\margin}:\radius)
				 	arc ({360/\n * (\s - 1)+\margin}:{360/\n * (\s)-\margin}:\radius);
			}
			\foreach \s in {3} {
				\node[draw, circle] at ({360/\n * (\s - 1)}:\radius) {$\s$};
			}
		\end{tikzpicture}
	\end{figure}
	
\end{frame}

%---------------------------------------------------------------------------------
%-------------------------------EXAMPLE 3---------------------------------
%---------------------------------------------------------------------------------

\begin{frame}
	\frametitle{Rotor walk example}
	
	\begin{figure}
		\begin{tikzpicture}
			\def \n {3}
			\def \radius {3cm}
			\def \margin {8}
			\foreach \s in {1} {
				\node[draw, circle] at ({360/\n * (\s - 1)}:\radius) {$\s$};
			}
			\foreach \s in {2} {
				\node[draw, circle, fill=black] at ({360/\n * (\s - 1)}:\radius) {\color{white} $\s$};
				\draw[<->, >=latex] ({360/\n * (\s - 1)+\margin}:\radius)
				 	arc ({360/\n * (\s - 1)+\margin}:{360/\n * (\s)-\margin}:\radius);
			}
			\foreach \s in {3} {
				\node[draw, circle] at ({360/\n * (\s - 1)}:\radius) {$\s$};
				\draw[<-, >=latex] ({360/\n * (\s - 1)+\margin}:\radius)
				 	arc ({360/\n * (\s - 1)+\margin}:{360/\n * (\s)-\margin}:\radius);
			}
		\end{tikzpicture}
	\end{figure}
	
\end{frame}

%---------------------------------------------------------------------------------
%-------------------------------EXAMPLE 4---------------------------------
%---------------------------------------------------------------------------------

\begin{frame}
	\frametitle{Rotor walk example}
	
	\begin{figure}
		\begin{tikzpicture}
			\def \n {3}
			\def \radius {3cm}
			\def \margin {8}
			\foreach \s in {1,2} {
				\node[draw, circle] at ({360/\n * (\s - 1)}:\radius) {$\s$};
				\draw[<-, >=latex] ({360/\n * (\s - 1)+\margin}:\radius)
				 	arc ({360/\n * (\s - 1)+\margin}:{360/\n * (\s)-\margin}:\radius);
			}
			\foreach \s in {3} {
				\node[draw, circle, fill=black] at ({360/\n * (\s - 1)}:\radius) {\color{white} $\s$};
				\draw[<-, >=latex] ({360/\n * (\s - 1)+\margin}:\radius)
				 	arc ({360/\n * (\s - 1)+\margin}:{360/\n * (\s)-\margin}:\radius);
			}
		\end{tikzpicture}
	\end{figure}
	
\end{frame}

%---------------------------------------------------------------------------------
%-------------------------------EXAMPLE 5---------------------------------
%---------------------------------------------------------------------------------

\begin{frame}
	\frametitle{Rotor walk example}
	
	\begin{figure}
		\begin{tikzpicture}
			\def \n {3}
			\def \radius {3cm}
			\def \margin {8}
			\foreach \s in {1} {
				\node[draw, circle] at ({360/\n * (\s - 1)}:\radius) {$\s$};
				\draw[<-, >=latex] ({360/\n * (\s - 1)+\margin}:\radius)
				 	arc ({360/\n * (\s - 1)+\margin}:{360/\n * (\s)-\margin}:\radius);
			}
			\foreach \s in {2} {
				\node[draw, circle, fill=black] at ({360/\n * (\s - 1)}:\radius) {\color{white} $\s$};
			}
			\foreach \s in {3} {
				\node[draw, circle] at ({360/\n * (\s - 1)}:\radius) {$\s$};
				\draw[<->, >=latex] ({360/\n * (\s - 1)+\margin}:\radius)
				 	arc ({360/\n * (\s - 1)+\margin}:{360/\n * (\s)-\margin}:\radius);
			}
		\end{tikzpicture}
	\end{figure}
	
\end{frame}

%---------------------------------------------------------------------------------
%-------------------------------EXAMPLE 6---------------------------------
%---------------------------------------------------------------------------------

\begin{frame}
	\frametitle{Rotor walk example}
	
	\begin{figure}
		\begin{tikzpicture}
			\def \n {3}
			\def \radius {3cm}
			\def \margin {8}
			\foreach \s in {1} {
				\node[draw, circle, fill=black] at ({360/\n * (\s - 1)}:\radius) {\color{white} $\s$};
			}
			\foreach \s in {2} {
				\node[draw, circle] at ({360/\n * (\s - 1)}:\radius) {$\s$};
				\draw[->, >=latex] ({360/\n * (\s - 1)+\margin}:\radius)
				 	arc ({360/\n * (\s - 1)+\margin}:{360/\n * (\s)-\margin}:\radius);
			}
			\foreach \s in {3} {
				\node[draw, circle] at ({360/\n * (\s - 1)}:\radius) {$\s$};
				\draw[<->, >=latex] ({360/\n * (\s - 1)+\margin}:\radius)
				 	arc ({360/\n * (\s - 1)+\margin}:{360/\n * (\s)-\margin}:\radius);
			}
		\end{tikzpicture}
	\end{figure}
	
\end{frame}

%---------------------------------------------------------------------------------
%-------------------------------EXAMPLE 7---------------------------------
%---------------------------------------------------------------------------------

\begin{frame}
	\frametitle{Rotor walk example}
	
	\begin{figure}
		\begin{tikzpicture}
			\def \n {3}
			\def \radius {3cm}
			\def \margin {8}
			\foreach \s in {1,2} {
				\node[draw, circle] at ({360/\n * (\s - 1)}:\radius) {$\s$};
				\draw[->, >=latex] ({360/\n * (\s - 1)+\margin}:\radius)
				 	arc ({360/\n * (\s - 1)+\margin}:{360/\n * (\s)-\margin}:\radius);
			}
			\foreach \s in {3} {
				\node[draw, circle, fill=black] at ({360/\n * (\s - 1)}:\radius) {\color{white} $\s$};
				\draw[->, >=latex] ({360/\n * (\s - 1)+\margin}:\radius)
				 	arc ({360/\n * (\s - 1)+\margin}:{360/\n * (\s)-\margin}:\radius);
			}
		\end{tikzpicture}
	\end{figure}
	
\end{frame}

%---------------------------------------------------------------------------------
%--------------------------------BLANK Z2-----------------------------------
%---------------------------------------------------------------------------------

%\begin{frame}
%	\frametitle{Rotor routing on $\mathbb{Z}^2$}
	
%\end{frame}

%---------------------------------------------------------------------------------
%----------------------------ROTORS ON Z2-------------------------------
%---------------------------------------------------------------------------------
 
 \begin{frame}
 	\frametitle{Rotor routing on $\mathbb{Z}^2$}
	
	\pause
	
	\begin{defn}[Rotor mechanism on $\mathbb{Z}^2$]
		There are four possible rotors starting at a given point: $$\mathcal{E} = \{\text{North}, \text{East}, \text{South}, \text{West} \}.$$ Rotors turn counter-clockwise, repeatedly cycling through $\mathcal{E}$.
	\end{defn}
 \pause
 \includegraphics[width=\textwidth]{Arrows}

 \end{frame}
 
%---------------------------------------------------------------------------------
%----------------------------SMALL Z2 DEMO-----------------------------
%---------------------------------------------------------------------------------
 
 % \begin{frame}
 %	\frametitle{Rotor routing on $\mathbb{Z}^2$}
	
%	\begin{center}
%		\includegraphics[width=\textwidth]{Arrows}
%	\end{center}
 
 %\end{frame}
 
%---------------------------------------------------------------------------------
%------------------------BLANK RECURRENCE-------------------------
%---------------------------------------------------------------------------------

%\begin{frame}
%	\frametitle{Recurrence and transience}
	
%\end{frame}

%---------------------------------------------------------------------------------
%--------------------RECURRENCE DEFINITION----------------------
%---------------------------------------------------------------------------------

\begin{frame}
	\frametitle{Recurrence and transience}
	\pause
	\begin{defn}[Recurrence]
		A walk is recurrent if it visits each point \underline{infinitely} many times.
	\end{defn}
	
	\pause 
	
	\begin{itemize}
		\item Random walk on $\mathbb{Z}$, $\mathbb{Z}^2$
		\item Repeated rotor walks on $\mathbb{Z}$, $\mathbb{Z}^2$ with single-direction configuration
	\end{itemize}
	
	\pause	  
	
	\begin{defn}[Transience]
		A walk is transient if it visits each point \underline{finitely} many times.
	\end{defn}
	
	\pause
	
	\begin{itemize}
		\item Random walk on $\mathbb{Z}^d$ where $d \ge 3$
		\item Repeated rotor walks on $\mathbb{Z}^3$ with single-direction configuration
	\end{itemize}
	  
%\end{frame}

%---------------------------------------------------------------------------------
%---------------------------BLANK EXAMPLES---------------------------
%---------------------------------------------------------------------------------

%\begin{frame}
%	\frametitle{Examples of recurrence and transience}
	
%\end{frame}

%---------------------------------------------------------------------------------
%---------------------RECURRENCE EXAMPLES----------------------
%---------------------------------------------------------------------------------

%\begin{frame}
%	\frametitle{Examples of recurrence and transience}
\end{frame}

%---------------------------------------------------------------------------------
%---------------------------BLANK THEOREMS---------------------------
%---------------------------------------------------------------------------------

\begin{frame}
	\frametitle{Theorems about recurrence and transience}
	
\end{frame}

%---------------------------------------------------------------------------------
%----------------------RECURRENCE THEOREM----------------------
%---------------------------------------------------------------------------------

\begin{frame}
	\frametitle{Theorems about recurrence and transience}
	
	\begin{theorem}[Angel-Holroyd 2011]
		Given a rotor mechanism and a configuration, a single rotor walk is either recurrent for every starting vertex, or transient for every starting vertex.
	\end{theorem}
	
%\end{frame}

%---------------------------------------------------------------------------------
%---------------------RECURRENCE COROLLARY--------------------
%---------------------------------------------------------------------------------
\pause
%\begin{frame}
	\frametitle{Theorems about recurrence and transience}
	
	\begin{cor}
		Given a rotor mechanism, two rotor configurations differing at only finitely many vertices are either both recurrent or both transient.
	\end{cor}
	
\end{frame}

%---------------------------------------------------------------------------------
%--------------------------COROLLARY PROOF-------------------------
%---------------------------------------------------------------------------------

\begin{frame}
	\frametitle{Theorems about recurrence and transience}
	
	\begin{cor}
		Given a rotor mechanism, two rotor configurations differing at only finitely many vertices are either both recurrent or both transient.
	\end{cor}
	
	\begin{proof}
		We need only consider two rotor configurations $r, r'$ differing at only one vertex $v$, such that a single rotation of the rotor at $v$ produces $r'$ from $r$. If $r$ is recurrent, a rotor walk started at $v$ produces $r'$ with the particle moved to some other vertex. By the aforementioned theorem, $r'$ must also be recurrent.
	\end{proof}
	
\end{frame}

%---------------------------------------------------------------------------------
%-------------------------BLANK CONJECTURES-----------------------
%---------------------------------------------------------------------------------

\begin{frame}
	\frametitle{Random configuration on $\mathbb{Z}^2$}

\end{frame}

%---------------------------------------------------------------------------------
%--------------------RANDOMNESS CONJECTURE------------------
%---------------------------------------------------------------------------------

\begin{frame}
	\frametitle{Random configuration on $\mathbb{Z}^2$}
	
	\begin{conj}[Levine-Peres 2015]
		Given a random configuration of rotors on $\mathbb{Z}^2$, a rotor walk on that configuration is recurrent with probability $1$.
	\end{conj}
	
\end{frame}

%---------------------------------------------------------------------------------
%-------------------------HITTING FREQUENCY------------------------
%---------------------------------------------------------------------------------

\begin{frame}
	\frametitle{Random configuration on $\mathbb{Z}^2$}
	
	\includegraphics[width=\textwidth]{Hitting}
	
\end{frame}

%---------------------------------------------------------------------------------
%------------------------2-MANIFOLD EXAMPLES----------------------
%---------------------------------------------------------------------------------

\begin{frame}
	\frametitle{Rotor routing on other 2-manifolds}
	
	\pause
	
	\begin{defn}[2-manifold]
		A topological space whose points all have open disks as neighborhoods.
	\end{defn}
	
	\pause
	
	\hspace{2cm}
	
	Some 2-manifolds:
	\begin{multicols}{2}
		\begin{itemize}
			\item Cylinder
			\item M{\"o}bius strip
			\item Torus
			\item Sphere
			\item Klein bottle
			\item Real projective plane
		\end{itemize}
	\end{multicols}

\end{frame}

%---------------------------------------------------------------------------------
%----------------------COMPACTNESS DEFINITION------------------
%---------------------------------------------------------------------------------
\begin{frame}
	\frametitle{Rotor routing on other 2-manifolds}
	
	\begin{defn}[Compactness]
		If a space is compact, every open cover has a finite sub-cover.
	\end{defn}
	
	\pause
	
	\begin{multicols}{2}
		Non-compact 2-manifolds:
		\begin{itemize}
			\item Plane
			\item Cylinder
			\item M{\"o}bius strip
		\end{itemize}
		
		\pause
	
		\hspace{2cm}
	
		Compact 2-manifolds:
		\begin{itemize}
			\item Torus
			\item Sphere
			\item Klein bottle
			\item Real projective plane
		\end{itemize}
	\end{multicols}
	
\end{frame}

%---------------------------------------------------------------------------------
%-------------------------------GLUING DIAGRAM-----------------------
%---------------------------------------------------------------------------------

\begin{frame}
	\frametitle{Rotor routing on other 2-manifolds}
	
	A gluing diagram shows how edges of a polygon are connected to create a surface. When a particle hits an edge that is associated to another, the particle "reappears" on the associated edge.
	\vspace{-0.5cm}
	\begin{center}
		\includegraphics[height=200px]{Glue}
	\end{center}

\end{frame}

%---------------------------------------------------------------------------------
%--------------------------BLANK CONJECTURES----------------------
%---------------------------------------------------------------------------------

\begin{frame}
	\frametitle{Experimental results}
	
\end{frame}

%---------------------------------------------------------------------------------
%---------------------------PLAIN CONJECTURE------------------------
%---------------------------------------------------------------------------------

\begin{frame}
	\frametitle{Experimental results}
	
	\begin{conj}
		Repeated rotor walks on the M{\"o}bius strip, cylinder, and plane will eventually create congruent octagonal patterns for the single-direction configuration.
	\end{conj}
	
%\end{frame}

%---------------------------------------------------------------------------------
%-------------------------OCTAGON CONJECTURES------------------
%---------------------------------------------------------------------------------

%\begin{frame}
%	\frametitle{Experimental results}
	
%	\begin{conj}
%		Every non-compact 2-manifold will eventually create a congruent octagonal pattern.
%	\end{conj}
\pause	
	\hspace{0.8cm}
	
	\includegraphics[width=\textwidth]{Octagon}

\end{frame}

%---------------------------------------------------------------------------------
%-----------------------COMPACT LOOP THEOREM------------------
%---------------------------------------------------------------------------------

\begin{frame}
	\frametitle{Experimental results}
	
	\begin{observe}
		For any rotor configuration and rotor mechanism, a rotor walk on a compact surface eventually results in a finite loop of configurations.
	\end{observe}

\end{frame}

%---------------------------------------------------------------------------------
%-------------------------------FUTURE WORK----------------------------
%---------------------------------------------------------------------------------

\begin{frame}
	\frametitle{Future work}
	
	\begin{itemize}
		\item Testing alternate configurations, including different lattices and transient configurations.
		\item Experimenting with rotor routing in higher dimensions.
		\item Developing a predictive model for surface type based on loop length on compact surfaces.
	\end{itemize}

\end{frame}
 
\end{document}